\documentclass[11pt,preprint, authoryear]{elsarticle}

\usepackage{lmodern}
%%%% My spacing
\usepackage{setspace}
\setstretch{1.2}
\DeclareMathSizes{12}{14}{10}{10}

% Wrap around which gives all figures included the [H] command, or places it "here". This can be tedious to code in Rmarkdown.
\usepackage{float}
\let\origfigure\figure
\let\endorigfigure\endfigure
\renewenvironment{figure}[1][2] {
    \expandafter\origfigure\expandafter[H]
} {
    \endorigfigure
}

\let\origtable\table
\let\endorigtable\endtable
\renewenvironment{table}[1][2] {
    \expandafter\origtable\expandafter[H]
} {
    \endorigtable
}


\usepackage{ifxetex,ifluatex}
\usepackage{fixltx2e} % provides \textsubscript
\ifnum 0\ifxetex 1\fi\ifluatex 1\fi=0 % if pdftex
  \usepackage[T1]{fontenc}
  \usepackage[utf8]{inputenc}
\else % if luatex or xelatex
  \ifxetex
    \usepackage{mathspec}
    \usepackage{xltxtra,xunicode}
  \else
    \usepackage{fontspec}
  \fi
  \defaultfontfeatures{Mapping=tex-text,Scale=MatchLowercase}
  \newcommand{\euro}{€}
\fi

\usepackage{amssymb, amsmath, amsthm, amsfonts}

\def\bibsection{\section*{References}} %%% Make "References" appear before bibliography


\usepackage[round]{natbib}

\usepackage{longtable}
\usepackage[margin=2.3cm,bottom=2cm,top=2.5cm, includefoot]{geometry}
\usepackage{fancyhdr}
\usepackage[bottom, hang, flushmargin]{footmisc}
\usepackage{graphicx}
\numberwithin{equation}{section}
\numberwithin{figure}{section}
\numberwithin{table}{section}
\setlength{\parindent}{0cm}
\setlength{\parskip}{1.3ex plus 0.5ex minus 0.3ex}
\usepackage{textcomp}
\renewcommand{\headrulewidth}{0.2pt}
\renewcommand{\footrulewidth}{0.3pt}

\usepackage{array}
\newcolumntype{x}[1]{>{\centering\arraybackslash\hspace{0pt}}p{#1}}

%%%%  Remove the "preprint submitted to" part. Don't worry about this either, it just looks better without it:
\makeatletter
\def\ps@pprintTitle{%
  \let\@oddhead\@empty
  \let\@evenhead\@empty
  \let\@oddfoot\@empty
  \let\@evenfoot\@oddfoot
}
\makeatother

 \def\tightlist{} % This allows for subbullets!

\usepackage{hyperref}
\hypersetup{breaklinks=true,
            bookmarks=true,
            colorlinks=true,
            citecolor=blue,
            urlcolor=blue,
            linkcolor=blue,
            pdfborder={0 0 0}}


% The following packages allow huxtable to work:
\usepackage{siunitx}
\usepackage{multirow}
\usepackage{hhline}
\usepackage{calc}
\usepackage{tabularx}
\usepackage{booktabs}
\usepackage{caption}


\newenvironment{columns}[1][]{}{}

\newenvironment{column}[1]{\begin{minipage}{#1}\ignorespaces}{%
\end{minipage}
\ifhmode\unskip\fi
\aftergroup\useignorespacesandallpars}

\def\useignorespacesandallpars#1\ignorespaces\fi{%
#1\fi\ignorespacesandallpars}

\makeatletter
\def\ignorespacesandallpars{%
  \@ifnextchar\par
    {\expandafter\ignorespacesandallpars\@gobble}%
    {}%
}
\makeatother

\newlength{\cslhangindent}
\setlength{\cslhangindent}{1.5em}
\newenvironment{CSLReferences}%
  {\setlength{\parindent}{0pt}%
  \everypar{\setlength{\hangindent}{\cslhangindent}}\ignorespaces}%
  {\par}


\urlstyle{same}  % don't use monospace font for urls
\setlength{\parindent}{0pt}
\setlength{\parskip}{6pt plus 2pt minus 1pt}
\setlength{\emergencystretch}{3em}  % prevent overfull lines
\setcounter{secnumdepth}{5}

%%% Use protect on footnotes to avoid problems with footnotes in titles
\let\rmarkdownfootnote\footnote%
\def\footnote{\protect\rmarkdownfootnote}
\IfFileExists{upquote.sty}{\usepackage{upquote}}{}

%%% Include extra packages specified by user

%%% Hard setting column skips for reports - this ensures greater consistency and control over the length settings in the document.
%% page layout
%% paragraphs
\setlength{\baselineskip}{12pt plus 0pt minus 0pt}
\setlength{\parskip}{12pt plus 0pt minus 0pt}
\setlength{\parindent}{0pt plus 0pt minus 0pt}
%% floats
\setlength{\floatsep}{12pt plus 0 pt minus 0pt}
\setlength{\textfloatsep}{20pt plus 0pt minus 0pt}
\setlength{\intextsep}{14pt plus 0pt minus 0pt}
\setlength{\dbltextfloatsep}{20pt plus 0pt minus 0pt}
\setlength{\dblfloatsep}{14pt plus 0pt minus 0pt}
%% maths
\setlength{\abovedisplayskip}{12pt plus 0pt minus 0pt}
\setlength{\belowdisplayskip}{12pt plus 0pt minus 0pt}
%% lists
\setlength{\topsep}{10pt plus 0pt minus 0pt}
\setlength{\partopsep}{3pt plus 0pt minus 0pt}
\setlength{\itemsep}{5pt plus 0pt minus 0pt}
\setlength{\labelsep}{8mm plus 0mm minus 0mm}
\setlength{\parsep}{\the\parskip}
\setlength{\listparindent}{\the\parindent}
%% verbatim
\setlength{\fboxsep}{5pt plus 0pt minus 0pt}



\begin{document}



\begin{frontmatter}  %

\title{Helping You Write Academic Papers in R using Texevier}

% Set to FALSE if wanting to remove title (for submission)




\author[Add1]{Nico Katzke\footnote{\textbf{Contributions:}
  \newline \emph{The authors would like to thank no institution for
  money donated to this project. Thank you sincerely.}}}
\ead{nfkatzke@gmail.com}

\author[Add1,Add2]{John Smith}
\ead{John@gmail.com}

\author[Add1,Add2]{John Doe}
\ead{Joe@gmail.com}



\address[Add1]{Prescient Securities, Cape Town, South Africa}
\address[Add2]{Some other Institution, Cape Town, South Africa}

\cortext[cor]{Corresponding author: Nico Katzke\footnote{\textbf{Contributions:}
  \newline \emph{The authors would like to thank no institution for
  money donated to this project. Thank you sincerely.}}}

\begin{abstract}
\small{
Abstract to be written here. The abstract should not be too long and
should provide the reader with a good understanding what you are writing
about. Academic papers are not like novels where you keep the reader in
suspense. To be effective in getting others to read your paper, be as
open and concise about your findings here as possible. Ideally, upon
reading your abstract, the reader should feel he / she must read your
paper in entirety.
}
\end{abstract}

\vspace{1cm}

\begin{keyword}
\footnotesize{
Multivariate GARCH \sep Kalman Filter \sep Copula \\ \vspace{0.3cm}
\textit{JEL classification} L250 \sep L100
}
\end{keyword}
\vspace{0.5cm}
\end{frontmatter}



%________________________
% Header and Footers
%%%%%%%%%%%%%%%%%%%%%%%%%%%%%%%%%
\pagestyle{fancy}
\chead{}
\rhead{}
\lfoot{}
\rfoot{\footnotesize Page \thepage}
\lhead{}
%\rfoot{\footnotesize Page \thepage } % "e.g. Page 2"
\cfoot{}

%\setlength\headheight{30pt}
%%%%%%%%%%%%%%%%%%%%%%%%%%%%%%%%%
%________________________

\headsep 35pt % So that header does not go over title




\hypertarget{introduction}{%
\section{\texorpdfstring{Introduction
\label{Introduction}}{Introduction }}\label{introduction}}

path dependence; first rule of investment management

Due to the aforementioned sensitivity issues surrounding errors in the
expected return estimation, this work will only cover the portfolio
optimisation techniques that intentionally forego this input. These
include the naive equal weight, inverse variance, hierarchical risk
parity, ERC and the minimum variance portfolios. \emph{The theoretical
underpinnings of each will be reviewed as well as their relative
performance in historical back tests}.

\hypertarget{aims-and-objectives}{%
\section{Aims and Objectives}\label{aims-and-objectives}}

This work aims to use Monte Carlo Methods to uncover the relationship
between a market's covariance structure and the risk return properties
of various portfolio optimisation algorithms. This will be achieved
through the following objectives.

\begin{enumerate}
\def\labelenumi{\arabic{enumi}.}
\item
  Design and create five distinct \emph{ad hoc} \emph{20 by 20}
  correlation matrices, each corresponding to a different risk structure
  or market type. These will range from a structure possessing no
  clusters to those possessing hierarchical clustering.
\item
  Use the R package \emph{MCmarket} to perform Monte Carlo Simulations
  for each of the five \emph{ad hoc} correlation matrices {[}REFERENCE
  MYSELF{]}. The markets will be designed to have student t multivariate
  distributions, with 3 degrees of freedom, the returns will each be
  univariate normally distributed with a periodic mean of 0.02 and
  standard deviation of 0.1. Each market type will be simulated 10 000
  times across 300 periods.
\item
  Use the simulated markets to calculate the returns obtained from the
  portfolio optimisers. The first 50 periods will be used estimate the
  \emph{ex ante} covariance matrix and the portfolio returns will be
  calculated on the remaining 250 periods. (should properly incorporate
  rebalancing!!!!!!! every 50 periods?). The
\item
  The performance of each portfolio optimiser will then be compared and
  contrasted using various portfolio risk/return analytics. Portfolio
  optimisers will be compared with each other within market types and
  with themselves across markets types.
\end{enumerate}

\hypertarget{litrature-review}{%
\section{Litrature Review}\label{litrature-review}}

\hypertarget{a-review-of-portfolio-optimisation-algorithms}{%
\subsection{A Review of Portfolio Optimisation
Algorithms}\label{a-review-of-portfolio-optimisation-algorithms}}

\hypertarget{introduction-1}{%
\subsubsection{Introduction}\label{introduction-1}}

Since Harry Markovitz's (1952) seminal work on the mean-variance
portfolios scholars from around the globe have been aspiring to develop
a robust algorithm capable of \emph{ex ante} situating a portfolio on
the efficient frontier (\protect\hyperlink{ref-markowitz}{Markowitz,
1952}). There are now a wide array of available alternatives portfolio
optimisers. They range from simple heuristic based approaches to
advanced mathematical algorithms based on quadratic optimisation, random
matrix theory and machine learning methods; many more are still in the
making.

These optimisers however are not without their flaws. Mean-variance
optimisers, in particular, rely heavily on the accuracy of return
forecasts, where small changes in the expected return input can lead to
large changes in portfolio weights (\protect\hyperlink{ref-lopez}{De
Prado, 2016}). Due to this issue only so-called risk based portfolio's
that intentionally avoid using expected return forecasts are discussed
in this work as they are said to be more robust to estimation error
(\protect\hyperlink{ref-maillard2010}{Maillard, 2010}). Furthermore,
quadratic programming methods used in portfolio optimisation require the
inversion of some positive-definite covariance matrix, which can be
susceptible to error if the covariance matrix suffers from a high
condition number. where a condition number is defined as the absolute
value of the ratio between a covariance matrix's largest and smallest
eigenvalues (\protect\hyperlink{ref-lopez2012}{Bailey \& Lopez De Prado,
2012}; \protect\hyperlink{ref-lopez}{De Prado, 2016}). Diagonal matrices
have the smallest condition number which increases as more correlated
variables are added. When working with high conditional number matrices
a small change in a single entry's estimated covariance can greatly
alter its inverse of, which in turn can effect the portfolio weights
(\protect\hyperlink{ref-lopez}{De Prado, 2016}). This is exacerbated by
the fact that covariance matrices themselves are prone to estimation
error. For a given sample size, larger dimension co-variance matrices
are prone to more noise in estimation. This is essentially due to a
reduction in degrees of freedom as a sample of at least \(1/2N(N+1)\)
independent and identically distributed (iid) observations are required
to estimate an \(N\times N\) covariance matrix. Furthermore, financial
market covariance structures tend to vary over time and have been know
to change rapidly during so-called regime changes
(\protect\hyperlink{ref-lopez}{De Prado, 2016}). \textbf{{[}Regime
changes{]}}

\hypertarget{naive-equal-weight-ew}{%
\subsubsection{Naive Equal Weight (EW)}\label{naive-equal-weight-ew}}

Perhaps the oldest and most simple portfolio diversification heuristic
constitutes holding a weight of \(1/N\) of the \(N\) total assets
available to the investor
(\protect\hyperlink{ref-demiguel2009}{DeMiguel, Garlappi \& Uppal,
2009}). This portfolio is commonly called the equal weight or 1/N
portfolio, its failure to recognise the importance of covariation
between assets has resulted in it also being referred to as the naive
portfolio. It commonly used as a benchmark index.

Despite its simplistic nature empirical studies tend to find a
statistically insignificant difference in Sharp ratio between the naive
portfolio and more advanced portfolio optimisers. This finding was made
in \protect\hyperlink{ref-demiguel2009}{DeMiguel, Garlappi \& Uppal}
(\protect\hyperlink{ref-demiguel2009}{2009}) who looked at the
mean-variance, minimum-variance and Bayes-Stein portfolio's, where EW
also performed surprisingly well form a total return perspective.

\hypertarget{minimum-variance-mv}{%
\subsubsection{Minimum Variance (MV)}\label{minimum-variance-mv}}

Portfolio optimisers designed to exhibit the minimum variance have
garnered a lot of attention for themselves, largely to their tendency to
achieve surprisingly high returns in historical back tests
(\protect\hyperlink{ref-clarke2011}{Clarke, De Silva \& Thorley, 2011}).
This performance has been attributed to the fact that low volatility
stocks tend to earn returns in excess of the market, and high beta
stocks tend not to be rewarded by higher returns
(\protect\hyperlink{ref-clarke2011}{Clarke, De Silva \& Thorley, 2011};
\protect\hyperlink{ref-fama1992}{Fama \& French, 1992}). These minimum
variance portfolios tend to achieve cumulative returns equal to or
slightly greater than market capitalization weighted portfolio's whilst
maintaining consistently lower variance and achieving a noticeable
improvement in downside risk mitigation even during times of financial
crisis (\protect\hyperlink{ref-clarke2011}{Clarke, De Silva \& Thorley,
2011}). \textbf{The MV portfolio (discussed in this section) is the only
portfolio on the efficient frontier that does not depend on expected
return forecasts(\protect\hyperlink{ref-lopez}{De Prado, 2016}).}

The minimum variance portfolio selects security weights such that the
resulting portfolio weights correspond to the portfolio with the lowest
possible in sample volatility. Therefore, it has the lowest expected
volatility and is, in theory, safest/least risky portfolio
(\protect\hyperlink{ref-rawl2012}{De Carvalho, Lu \& Moulin, 2012a}).
Its primary input is a variance covariance matrix, which it uses in its
optimisation to overweight low volatility and low correlation securities
(\protect\hyperlink{ref-rawl2012}{De Carvalho, Lu \& Moulin, 2012a}).
This approach often works well out of sample, but is known to achieve
minor reductions in \emph{ex ante} portfolio volatility by greatly
favouring a small number of low volatility/correlation securities
(\protect\hyperlink{ref-lopez}{De Prado, 2016} {[}p.~68{]}). This
tendency to produce highly concentrated portfolio's can cause serious
out of sample diversification problem as the sample becomes increasingly
susceptible to measurement error.

Let \(\sum\) indicate the markets variance covariance matrix and
\(w=\{w_i,..., w_N \}\) be a vector of length N containing individual
security weights, then the vector containing MV portfolio can be written
as (\protect\hyperlink{ref-rawl2012}{De Carvalho, Lu \& Moulin, 2012a}):

\(w^*=arg\min(w'\sum w)\ \ \ s.t.\ \sum^N_iw_i=1\)

\hypertarget{inverse-varience-iv-weighting}{%
\subsubsection{Inverse-Varience (IV)
Weighting}\label{inverse-varience-iv-weighting}}

\textbf{The inverse-variance (IV), equal risk contrition (ERC) and
maximum diversification (MD) portfolio's each assume that adequate
diversification can be obtained by allocating equal risk to each
investable security.}

The IV portfolio, also known as the equal-risk budget (ERB), portfolio
aims to allocate an equal risk budget to each investable security
(\protect\hyperlink{ref-leote}{De Carvalho, Lu \& Moulin, 2012b}). Where
the risk budget is defined as the the product of a the security's weight
and volatility. Therefore, if we define \(\sigma_i\) as security i's
volatility, then marginal volatility is equally distributed across N
securities by setting security weights as such:

\(w_{iv}=(\frac{1/\sigma_1}{\sum^N_{j=1} 1/\sigma}, ...,\frac{1/\sigma_N}{\sum^N_{j=1} 1/\sigma} )\)

\hypertarget{equal-risk-contribution-erc}{%
\subsubsection{Equal Risk Contribution
(ERC)}\label{equal-risk-contribution-erc}}

The ERC portfolio is similar to the IV, but also takes covariance into
account (\protect\hyperlink{ref-leote}{De Carvalho, Lu \& Moulin,
2012b}). The basic idea behind the ERC is to weight the portfolio such
that each security contributes equally to risk, which in turn maximises
risk diversification (\protect\hyperlink{ref-maillard2010}{Maillard,
2010}). Generally speaking the ERC acts similar to a weight constrained
minimum variance portfolio, with constraints ensuring that adequate
diversification is maintained. The weights of the ERC portfolio
\(x=(x_1,x_2,...,x_n)\) consisting of n assets is calculated as follows.

let \(\sigma_i^2\) resemble asset i's variance, \(\sigma_{ij}\) the
covariance between asset i and j and \(\sum\) be the markets variance
covariance matrix. Portfolio risk can now be written as
\(sigma(x)=\sqrt{x^T\sum x}=\sum_i\sum_{j\neq i}x_ix_j\sigma_{ij}\)
(\protect\hyperlink{ref-maillard2010}{Maillard, 2010}). The marginal
risk contribution \(\partial_{x_i}\sigma(x)\) can then be defined as
follows (\protect\hyperlink{ref-maillard2010}{Maillard, 2010}):

\(\partial_{x_i}\sigma(x)=\frac{\partial\sigma(x)}{\partial x_i}=\frac{x_i\sigma_i^2+\sum_{j\neq i}x_j\sigma_{ij}}{\sigma(x)}\)

Therefore, \(\partial_{x_i}\sigma(x)\) refers to the change in portfolio
volatility resulting from a small change in asset i's weight. ERC uses
this definition to guide its algorithms central objective to equate the
risk contribution for each asset in the portfolio \emph{ex ante}. If we
define \((\sum x)_i\) as the \(i^{th}\) row resulting from the product
of \(\sum\) with x and note that \(\partial_{x_i}\sigma(x)=(\sum x)_i\),
then the optimal ERC weight can be written as
(\protect\hyperlink{ref-maillard2010}{Maillard, 2010}):

\(x^*=\{x \ \epsilon[0,1]^n:\sum x_i=1, x_i \times (\sum x)_i=x_j \times (\sum x)_j \ \forall \ i,j \}\)

\protect\hyperlink{ref-choueifaty2013}{Choueifaty, Froidure \& Reynier}
(\protect\hyperlink{ref-choueifaty2013}{2013}) conducted an empirical
back test comparing the relative performance if numerous portfolio
optimisers between 1999 and 2010. They used historical data from the
MSCI World world index and considered the largest 50\% of assets at each
semi-annual rebalance date. At each rebalance date the covariance
matrices, used as inputs in the portfolio optimisers, were estimated
using the passed years worth of data
(\protect\hyperlink{ref-choueifaty2013}{Choueifaty, Froidure \& Reynier,
2013}). This was done to reduce noise in estimation. All portfolio's
were restricted to long only. The MV portfolio achieved an annual return
of 6.7\% and outperformed the ERC and EW portfolio's who returned 6.3\%
and 5.8\% respectively. Unsurprisingly, the MV portfolio possessed the
lowest daily volatility (10\%) followed by the ERC and then the EW
portfolio's (with 12.9\% and 16.4\% respectively). Accordingly the MV
portfolio scored the highest sharp ratio (0.36) followed by the ERC and
EW portfolio's (0.24 and 0.16 respectively).

\hypertarget{maximum-diversification-md}{%
\subsubsection{Maximum Diversification
(MD)}\label{maximum-diversification-md}}

`The MD strategy, introduced by Choueifaty and Coignard {[}2008{]},
invests in the portfolio that maximizes a diversification ratio. The
ratio is the sum of the risk budget allocated to each stock in the
portfolio divided by the portfolio volatility. This strategy should
invest in stocks that are less correlated to other stocks.' (copy
pasted)

\hypertarget{hierarchical-risk-parity-hrp}{%
\subsubsection{Hierarchical Risk Parity
(HRP)}\label{hierarchical-risk-parity-hrp}}

Due to the multitude of robustness issues related to traditional
portfolio optimisers, \protect\hyperlink{ref-lopez}{De Prado}
(\protect\hyperlink{ref-lopez}{2016}) developed a new approach
incorporating machine-learning methods and graph theory
(\protect\hyperlink{ref-arevalo}{\textbf{arevalo?}}).
\protect\hyperlink{ref-lopez}{De Prado}
(\protect\hyperlink{ref-lopez}{2016}) argues that the ``lack of
hierarchical structure in a correlation matrix allows weights to vary
freely in unintended ways'' and that this contributes to the instability
issues. His HRP algorithm requires only a singular covariance matrix and
can utilize the information within without the need for the positive
definite property (\protect\hyperlink{ref-lopez}{De Prado, 2016}). This
procedure works in three stages:

\protect\hyperlink{ref-lopez}{De Prado}
(\protect\hyperlink{ref-lopez}{2016}) carried out an in sample
simulation study comparing the respective allocations of the long-only
minimum variance, IVP and HRP portfolios using a covariance matrix using
a condition number that is ``not unfavourable'' to the minimum variance
portfolio. The simulated data consisted of 10000 observations across 10
variables. The following findings were made: The minimum variance
portfolio concentrated 92.66\% of funds in the top 5 holdings and
assigned a zero weight to 3 assets. \emph{Conversly}, HRP only assigned
62.5\% of its funds to the top 5 holdings
(\protect\hyperlink{ref-lopez}{De Prado, 2016}). The minimum variance
portfolio's objective function causes it to build highly concentrated
portfolio's in favour of a small reduction in volatility; the HRP
portfolio had only a slightly higher volatility
(\protect\hyperlink{ref-lopez}{De Prado, 2016}). This apparent
diversification advantage achieved by the minimum variance portfolio is
rather deceptive as the portfolio remains highly susceptible to
idiosyncratic risk incidents within its top holdings
(\protect\hyperlink{ref-lopez}{De Prado, 2016}). This claim was further
validated by the finding that HRP achieved significantly lower out of
sample variance compared to the minimum variance portfolio.

``. Markowitz's curse is that the more correlated investments are, the
greater is the need for a diversified portfolio---and yet the greater
are that portfolio's estimation errors.
(\protect\hyperlink{ref-lopez}{De Prado, 2016})''

\hypertarget{inverse-variance-portfolio-ivp}{%
\subsubsection{Inverse Variance Portfolio
(IVP)}\label{inverse-variance-portfolio-ivp}}

\hypertarget{methadology}{%
\section{Methadology}\label{methadology}}

\hypertarget{results-and-discussion}{%
\section{Results and Discussion}\label{results-and-discussion}}

\hypertarget{conclusion}{%
\section{Conclusion}\label{conclusion}}

I hope you find this template useful. Remember, stackoverflow is your
friend - use it to find answers to questions. Feel free to write me a
mail if you have any questions regarding the use of this package. To
cite this package, simply type citation(``Texevier'') in Rstudio to get
the citation for \protect\hyperlink{ref-Texevier}{Katzke}
(\protect\hyperlink{ref-Texevier}{2017}) (Note that uncited references
in your bibtex file will not be included in References).

\newpage

\hypertarget{references}{%
\section*{References}\label{references}}
\addcontentsline{toc}{section}{References}

\textless divid=``refs''\textgreater{}

\newpage

\hypertarget{appendix}{%
\section*{Appendix}\label{appendix}}
\addcontentsline{toc}{section}{Appendix}

\hypertarget{appendix-a}{%
\subsection*{Appendix A}\label{appendix-a}}
\addcontentsline{toc}{subsection}{Appendix A}

Some appendix information here

\hypertarget{appendix-b}{%
\subsection*{Appendix B}\label{appendix-b}}
\addcontentsline{toc}{subsection}{Appendix B}

\hypertarget{refs}{}
\begin{CSLReferences}{1}{0}
\leavevmode\hypertarget{ref-lopez2012}{}%
Bailey, D.H. \& Lopez De Prado, M. 2012. Balanced baskets: A new
approach to trading and hedging risks. \emph{Journal of Investment
Strategies (Risk Journals)}. 1(4).

\leavevmode\hypertarget{ref-choueifaty2013}{}%
Choueifaty, Y., Froidure, T. \& Reynier, J. 2013. Properties of the most
diversified portfolio. \emph{Journal of investment strategies}.
2(2):49--70.

\leavevmode\hypertarget{ref-clarke2011}{}%
Clarke, R., De Silva, H. \& Thorley, S. 2011. Minimum-variance portfolio
composition. \emph{The Journal of Portfolio Management}. 37(2):31--45.

\leavevmode\hypertarget{ref-leote}{}%
De Carvalho, R.L., Lu, X. \& Moulin, P. 2012b. Demystifying equity
risk--based strategies: A simple alpha plus beta description. \emph{The
Journal of Portfolio Management}. 38(3):56--70.

\leavevmode\hypertarget{ref-rawl2012}{}%
De Carvalho, R.L., Lu, X. \& Moulin, P. 2012a. Demystifying equity
risk--based strategies: A simple alpha plus beta description. \emph{The
Journal of Portfolio Management}. 38(3):56--70.

\leavevmode\hypertarget{ref-lopez}{}%
De Prado, M.L. 2016. Building diversified portfolios that outperform out
of sample. \emph{The Journal of Portfolio Management}. 42(4):59--69.

\leavevmode\hypertarget{ref-demiguel2009}{}%
DeMiguel, V., Garlappi, L. \& Uppal, R. 2009. Optimal versus naive
diversification: How inefficient is the 1/n portfolio strategy?
\emph{The review of Financial studies}. 22(5):1915--1953.

\leavevmode\hypertarget{ref-fama1992}{}%
Fama, E.F. \& French, K.R. 1992. The cross-section of expected stock
returns. \emph{the Journal of Finance}. 47(2):427--465.

\leavevmode\hypertarget{ref-Texevier}{}%
Katzke, N.F. 2017. \emph{{Texevier}: {P}ackage to create elsevier
templates for rmarkdown}. Stellenbosch, South Africa: Bureau for
Economic Research.

\leavevmode\hypertarget{ref-maillard2010}{}%
Maillard, T., Roncalli. 2010. The properties of equally weighted risk
contribution portfolios. \emph{The Journal of Portfolio Management}.
36(4):60--70.

\leavevmode\hypertarget{ref-markowitz}{}%
Markowitz, H. 1952. Portfolio selection. \emph{The Journal of Finance}.
7(1):77--91.

\end{CSLReferences}

\bibliography{Tex/ref}





\end{document}
