\documentclass[11pt,preprint, authoryear]{elsarticle}

\usepackage{lmodern}
%%%% My spacing
\usepackage{setspace}
\setstretch{1.2}
\DeclareMathSizes{12}{14}{10}{10}

% Wrap around which gives all figures included the [H] command, or places it "here". This can be tedious to code in Rmarkdown.
\usepackage{float}
\let\origfigure\figure
\let\endorigfigure\endfigure
\renewenvironment{figure}[1][2] {
    \expandafter\origfigure\expandafter[H]
} {
    \endorigfigure
}

\let\origtable\table
\let\endorigtable\endtable
\renewenvironment{table}[1][2] {
    \expandafter\origtable\expandafter[H]
} {
    \endorigtable
}


\usepackage{ifxetex,ifluatex}
\usepackage{fixltx2e} % provides \textsubscript
\ifnum 0\ifxetex 1\fi\ifluatex 1\fi=0 % if pdftex
  \usepackage[T1]{fontenc}
  \usepackage[utf8]{inputenc}
\else % if luatex or xelatex
  \ifxetex
    \usepackage{mathspec}
    \usepackage{xltxtra,xunicode}
  \else
    \usepackage{fontspec}
  \fi
  \defaultfontfeatures{Mapping=tex-text,Scale=MatchLowercase}
  \newcommand{\euro}{€}
\fi

\usepackage{amssymb, amsmath, amsthm, amsfonts}

\def\bibsection{\section*{References}} %%% Make "References" appear before bibliography


\usepackage[round]{natbib}

\usepackage{longtable}
\usepackage[margin=2.3cm,bottom=2cm,top=2.5cm, includefoot]{geometry}
\usepackage{fancyhdr}
\usepackage[bottom, hang, flushmargin]{footmisc}
\usepackage{graphicx}
\numberwithin{equation}{section}
\numberwithin{figure}{section}
\numberwithin{table}{section}
\setlength{\parindent}{0cm}
\setlength{\parskip}{1.3ex plus 0.5ex minus 0.3ex}
\usepackage{textcomp}
\renewcommand{\headrulewidth}{0.2pt}
\renewcommand{\footrulewidth}{0.3pt}

\usepackage{array}
\newcolumntype{x}[1]{>{\centering\arraybackslash\hspace{0pt}}p{#1}}

%%%%  Remove the "preprint submitted to" part. Don't worry about this either, it just looks better without it:
\makeatletter
\def\ps@pprintTitle{%
  \let\@oddhead\@empty
  \let\@evenhead\@empty
  \let\@oddfoot\@empty
  \let\@evenfoot\@oddfoot
}
\makeatother

 \def\tightlist{} % This allows for subbullets!

\usepackage{hyperref}
\hypersetup{breaklinks=true,
            bookmarks=true,
            colorlinks=true,
            citecolor=blue,
            urlcolor=blue,
            linkcolor=blue,
            pdfborder={0 0 0}}


% The following packages allow huxtable to work:
\usepackage{siunitx}
\usepackage{multirow}
\usepackage{hhline}
\usepackage{calc}
\usepackage{tabularx}
\usepackage{booktabs}
\usepackage{caption}


\newenvironment{columns}[1][]{}{}

\newenvironment{column}[1]{\begin{minipage}{#1}\ignorespaces}{%
\end{minipage}
\ifhmode\unskip\fi
\aftergroup\useignorespacesandallpars}

\def\useignorespacesandallpars#1\ignorespaces\fi{%
#1\fi\ignorespacesandallpars}

\makeatletter
\def\ignorespacesandallpars{%
  \@ifnextchar\par
    {\expandafter\ignorespacesandallpars\@gobble}%
    {}%
}
\makeatother

\newlength{\cslhangindent}
\setlength{\cslhangindent}{1.5em}
\newenvironment{CSLReferences}%
  {\setlength{\parindent}{0pt}%
  \everypar{\setlength{\hangindent}{\cslhangindent}}\ignorespaces}%
  {\par}


\urlstyle{same}  % don't use monospace font for urls
\setlength{\parindent}{0pt}
\setlength{\parskip}{6pt plus 2pt minus 1pt}
\setlength{\emergencystretch}{3em}  % prevent overfull lines
\setcounter{secnumdepth}{5}

%%% Use protect on footnotes to avoid problems with footnotes in titles
\let\rmarkdownfootnote\footnote%
\def\footnote{\protect\rmarkdownfootnote}
\IfFileExists{upquote.sty}{\usepackage{upquote}}{}

%%% Include extra packages specified by user
\usepackage{amsmath}

%%% Hard setting column skips for reports - this ensures greater consistency and control over the length settings in the document.
%% page layout
%% paragraphs
\setlength{\baselineskip}{12pt plus 0pt minus 0pt}
\setlength{\parskip}{12pt plus 0pt minus 0pt}
\setlength{\parindent}{0pt plus 0pt minus 0pt}
%% floats
\setlength{\floatsep}{12pt plus 0 pt minus 0pt}
\setlength{\textfloatsep}{20pt plus 0pt minus 0pt}
\setlength{\intextsep}{14pt plus 0pt minus 0pt}
\setlength{\dbltextfloatsep}{20pt plus 0pt minus 0pt}
\setlength{\dblfloatsep}{14pt plus 0pt minus 0pt}
%% maths
\setlength{\abovedisplayskip}{12pt plus 0pt minus 0pt}
\setlength{\belowdisplayskip}{12pt plus 0pt minus 0pt}
%% lists
\setlength{\topsep}{10pt plus 0pt minus 0pt}
\setlength{\partopsep}{3pt plus 0pt minus 0pt}
\setlength{\itemsep}{5pt plus 0pt minus 0pt}
\setlength{\labelsep}{8mm plus 0mm minus 0mm}
\setlength{\parsep}{\the\parskip}
\setlength{\listparindent}{\the\parindent}
%% verbatim
\setlength{\fboxsep}{5pt plus 0pt minus 0pt}



\begin{document}



\begin{frontmatter}  %

\title{The Link Between Market Covariance Structure and the Performance of
Risk-Based Portfolios}

% Set to FALSE if wanting to remove title (for submission)




\author[Add1]{Nathan Potgieter\footnote{\textbf{Contributions:} \newline \emph{The
  author would like to thank Nico Katzke for helping me puzzle and prod
  my to to the eventual completion of this research project.}}}
\ead{19959672@sun.ac.za}





\address[Add1]{Stellenbosch University, Stellenbosch, South Africa}


\begin{abstract}
\small{
This work uses Monte Carlo methods to design and simulate financial
market returns for five distinctive markets types, with each market type
relating to a unique covariance structure. The equal weight, minimum
variance, inverse variance, equal risk contribution and maximum
diversification portfolios are each back tested in the simulated markets
and the relationship between the portfolio return characteristics and
the market covariance structure is evaluated. \textbf{FINDINGS}
}
\end{abstract}

\vspace{1cm}

\begin{keyword}
\footnotesize{
Monte Carlo \sep Risk-based Portfolios \sep Portfolio Selection
\sep Copula \\ \vspace{0.3cm}
\textit{JEL classification} L250 \sep L100
}
\end{keyword}
\vspace{0.5cm}
\end{frontmatter}



%________________________
% Header and Footers
%%%%%%%%%%%%%%%%%%%%%%%%%%%%%%%%%
\pagestyle{fancy}
\chead{}
\rhead{}
\lfoot{}
\rfoot{\footnotesize Page \thepage}
\lhead{}
%\rfoot{\footnotesize Page \thepage } % "e.g. Page 2"
\cfoot{}

%\setlength\headheight{30pt}
%%%%%%%%%%%%%%%%%%%%%%%%%%%%%%%%%
%________________________

\headsep 35pt % So that header does not go over title




\hypertarget{introduction}{%
\section{\texorpdfstring{Introduction
\label{Introduction}}{Introduction }}\label{introduction}}

Since Harry Markovitz's (1952) seminal work on mean-variance portfolios
scholars from around the globe have been aspiring to develop a robust
algorithm capable of situating a portfolio on the efficient frontier
\emph{ex ante}. There are now a wide array of available alternatives
raging from simple heuristic based approaches to advanced mathematical
algorithms based on quadratic optimization, random matrix theory and
machine learning methods; with many more are still in the making.

Unfortunately, portfolio optimisers of the mean-variance type suffer
from sensitivity issues, where even slight changes in their expected
return input cause large changes in optimal portfolio weights. This is
exacerbated by the fact that expected returns are notoriously difficult,
if not impossible, to accurately forecast (De Prado,
\protect\hyperlink{ref-lopez}{2016}). Due to this issue, this work
focuses solely on the so-called risk-based portfolio, defined by De
Carvalho \emph{et al.}
(\protect\hyperlink{ref-leote}{2012}\protect\hyperlink{ref-leote}{a}) as
``systemic quantitative approaches to portfolio allocation'' that solely
rely on views of risk when allocating capital. These strategies
therefore, do not require expected return forecasts and are said to be
more robust to estimation error. Despite their sole focus on risk
mitigation risk-based portfolio's often perform surprising well, form a
total return standpoint, in empirical back tests (Choueifaty \emph{et
al.}, \protect\hyperlink{ref-choueifaty2013}{2013}).

Furthermore, instead of opting for the standard empirical approach of
evaluating said strategies through the use of historical back tests,
this work opts to use Monte Carlo simulation methods to investigate the
link between the markets covairiance structure and portfolio
performance. Monte Carlo methods prove to be invaluable in answering
this question as they allow for the creation of \emph{ad hoc} markets
with predetermined risk return characteristics, and hence the researcher
is left with no uncertainty regarding the composition of the market.
This creates an environment ideal for experimentation as the researcher
has control over the market and can therefore adjust the independent
variable, in this case the markets correlation structure, and observe
the response in the dependent variable, which in this work are the
portfolio return characteristics.

The risk-based portfolios evaluated in this work include the naive equal
weight (EW), minimum variance (MV), inverse variance (IV), equal risk
contribution (ERC) and the maximum diversification (MD) portfolios.
Section \ref{aims} lays out this works aims and objects, Section
\ref{lit} provides a review of the relevant literature. This includes
some general issues plaguing the field of portfolio optimization, the
rational and theoretical underpinnings behind the five risk-based
portfolios, their relative performance in empirical back tests and
finally the importance of using Monte Carlo methods in finance. Section
\ref{methadology} discusses the methodology used to uncover the
relationship of interest, Section \ref{reasults} provides the discussion
and results, and finally Section \ref{conclusion} concludes.

\hypertarget{aims-and-objectives}{%
\section{\texorpdfstring{Aims and Objectives
\label{aims}}{Aims and Objectives }}\label{aims-and-objectives}}

This work aims to use Monte Carlo Methods to uncover the relationship
between a market's correlation structure and the risk return properties
of various risk-based portfolio algorithms. This is achieved through the
following objectives:

\begin{enumerate}
\def\labelenumi{\arabic{enumi}.}
\item
  Design and create four distinctive \emph{ad hoc} correlation matrices
  and estimate one empirical \emph{50 by 50} correlation matrix, each
  representing a unique market type. These will range from markets
  structures possessing no clusters to those exhibiting hierarchical
  clustering. All other risk characteristic will remain equal between
  market types.
\item
  Use the R package \emph{MCmarket} to perform Monte Carlo Simulations,
  with each of the five correlation matrices from step one acting as the
  primary input for their own Monte Carlo (Potgieter,
  \protect\hyperlink{ref-MCmarket}{2020}). The markets will be built to
  posses student t multivariate distributions, with 4.5 degrees of
  freedom. Meanwhile the individual asset returns will each be normally
  distributed wit their mean's and standard deviation's calibrated using
  S\&P500 data. Each of the five market types will be simulated 10 000
  times across 300 periods.
\item
  Use the simulated market data to calculate the returns obtained from
  various risk based portfolio's. The first \textbf{100} periods will be
  used estimate an out of sample covariance matrix, this will be used to
  calculate portfolio weights. These weights will remain for the next 50
  periods after which portfolios will be rebalanced by looking back
  \emph{100} periods, recalculating the covariance matrix and the new
  portfolio weights. This process is repeated until all \emph{300}
  simulated periods have been considered. Therefore, each portfolio will
  end up with a series of \emph{200} returns.
\item
  The performance of each portfolio will then be compared and contrasted
  using various portfolio risk/return analytics. Portfolio optimisers
  will be compared with each other within market types and with
  themselves across markets types.
\end{enumerate}

\hypertarget{litrature-review}{%
\section{\texorpdfstring{Litrature Review
\label{lit}}{Litrature Review }}\label{litrature-review}}

\hypertarget{a-review-of-portfolio-optimisation-algorithms}{%
\subsection{A Review of Portfolio Optimisation
Algorithms}\label{a-review-of-portfolio-optimisation-algorithms}}

\hypertarget{introduction-1}{%
\subsubsection{Introduction}\label{introduction-1}}

This literature review will cover some common issues discussed within
the literature surrounding portfolio optimization in general, the five
risk-based portfolios evaluated in this work, their respective
performance in both empirical back tests and Monte Carlo studies and
finally the importance of using Monte Carlo methods within the field of
finance.

\hypertarget{common-issues-portfolio-optimizers}{%
\subsubsection{Common Issues Portfolio
Optimizers}\label{common-issues-portfolio-optimizers}}

When operating in sample, Portfolio optimization tends to be a perfect
science, but out of sample it becomes more of an art form where it is
often preferable to use heuristic over hard rules. This section
highlights some general issues, highlighted within the portfolio
optimization literature, that tend to worsen their performance out of
sample.

Firstly, mean-variance optimisers, like those introduced by Markowitz
(\protect\hyperlink{ref-markowitz}{1952}), rely heavily on the accuracy
of their expected return forecasts. Small changes in their expected
return input can lead to large changes in portfolio weights (De Prado,
\protect\hyperlink{ref-lopez}{2016}). Since in practice expected returns
are extremely difficult to estimate accurately, this issue serves as a
major hindrance to their wide spread use. Due to this issue the
so-called risk based portfolio's that intentionally avoid using expected
return forecasts have garnered a lot of attention (Maillard,
\protect\hyperlink{ref-maillard2010}{2010}).

Unfortunately, these risk based portfolios are not void of issues. The
quadratic programming methods used in many portfolio optimisers,
including the mean variance and many risk-based, require the inversion
of some positive-definite covariance matrix. This positive definiteness
requirement can cause issues as covariance matrices estimated on
empirical data are sometimes not positive definite, in which case their
inverse does not exist and these portfolios don't have solutions (De
Prado, \protect\hyperlink{ref-lopez}{2016}). A common method to get
around this issue is to simply compute the nearest positive definite
matrix and use that instead (Bates \& Maechler,
\protect\hyperlink{ref-Matrix}{2019}; Higham,
\protect\hyperlink{ref-higham2002}{2002}).

The covariance estimation step is particularly susceptible to error if
the covariance matrix suffers from a high condition number. A condition
number is defined as the absolute value of the ratio between a
covariance matrix's largest and smallest eigenvalues (Bailey \& Lopez De
Prado, \protect\hyperlink{ref-lopez2012}{2012}; De Prado,
\protect\hyperlink{ref-lopez}{2016}). The condition number is smallest
in diagonal matrices (they have a condition number of 1) and increases
as more correlated variables are added. When working with high condition
number matrices a small change in a single entry's estimated covariance
can greatly alter its inverse, which in turn can effect the portfolio
weights (De Prado, \protect\hyperlink{ref-lopez}{2016}). This is related
to Markowitz's curse which De Prado
(\protect\hyperlink{ref-lopez}{2016}) summerised as ``the more
correlated investments are, the greater is the need for a diversified
portfolio---and yet the greater are that portfolio's estimation
errors''. Therefore, variance matrices are are prone to estimation error
(Zhou \emph{et al.}, \protect\hyperlink{ref-zhou2019}{2019}). For a
sample with a given number of periods, larger dimension covariance
matrices are prone to more noise in estimation. This is essentially due
to a reduction in degrees of freedom as a sample of at least
\(1/2N(N+1)\) independent and identically distributed (iid) observations
are required to estimate an \(N\times N\) covariance matrix (De Prado,
\protect\hyperlink{ref-lopez}{2016}: 60){]}. Furthermore, financial
market covariance structures tend to vary over time and have been know
to change rapidly during so-called regime changes (De Prado,
\protect\hyperlink{ref-lopez}{2016}). This exacerbates the issue of
requiring a large number of observations when estimating the covariance
matrix, as there is no guarantee that passed data will be a good
refection of the future and looking further into the passed decreases
the likelihood of it being so.

\hypertarget{risk-based-portfolios}{%
\subsubsection{Risk Based Portfolio's}\label{risk-based-portfolios}}

This section reviews the intuition and technical underpinnings within
the litrature surrounding the so-called risk-based portfolios. These
include the equal weight (EW), minimum variance (MV), inverse volatility
(IV), equal risk contribution (ERC) and the maximum diversification (MD)
portfolios. The EW is a simple heuristic approach, the minimum variance
is more akin to a Markovitz (1952) mean variance portfolio, while the
inverse-variance (IV), equal risk contrition (ERC) and maximum
diversification (MD) are quite similar in that they all assume that
adequate diversification can be obtained by allocating equal risk to
each investible security.

\hypertarget{naive-equal-weight-ew}{%
\paragraph{Naive Equal Weight (EW)}\label{naive-equal-weight-ew}}

Perhaps the oldest and most simple portfolio diversification heuristic
constitutes holding a weight of \(1/N\) of the \(N\) total assets
available to the investor (DeMiguel \emph{et al.},
\protect\hyperlink{ref-demiguel2009}{2009}). This strategy doesn't
require any data when allocating capital and doesn't involve any form of
optimization (DeMiguel \emph{et al.},
\protect\hyperlink{ref-demiguel2009}{2009}). In layman's terms this
strategy can be described as putting an equal number of eggs in each
available basket. This portfolio is commonly called the equal weight or
1/N portfolio, but its failure to recognize the importance of both the
asset variance and the covariance between assets has resulted in it also
being referred to as the naive portfolio. Meanwhile its simplicity means
that it has been widely used as a benchmark. From a mean variance
perspective equal weighting is optimal when there is no correlation
between securities and each possesses the same variance. In this
scenario, the EW is equivalent to the MV portfolio.

\hypertarget{minimum-variance-mv}{%
\paragraph{Minimum Variance (MV)}\label{minimum-variance-mv}}

Portfolio optimisers designed to exhibit the minimum variance have in
recent years garnered a lot of attention, largely due their tendency to
achieve surprisingly high returns in historical back tests (Clarke
\emph{et al.}, \protect\hyperlink{ref-clarke2011}{2011}). This
performance has been attributed to the empirical phenomena that low
volatility stocks tend to earn returns in excess of the market, and high
beta stocks tend not to be rewarded by higher returns (Clarke \emph{et
al.}, \protect\hyperlink{ref-clarke2011}{2011}; Fama \& French,
\protect\hyperlink{ref-fama1992}{1992}). These findings are contrary to
financial economic theory. The most widely cited of these portfolios,
the minimum variance (MV) tends to achieve cumulative returns equal to
or slightly greater than market capitalization weighted portfolio's
whilst maintaining consistently lower variance and achieving a
noticeable improvement in downside risk mitigation (Clarke \emph{et
al.}, \protect\hyperlink{ref-clarke2011}{2011}). Interestingly, the MV
portfolio is the only portfolio of the efficient frontier that does not
depend on expected return forecasts(De Prado,
\protect\hyperlink{ref-lopez}{2016}).

The minimum variance portfolio selects security weights such that the
resulting portfolio corresponds to that with the lowest possible in
sample volatility. Therefore, it has the lowest expected volatility and
is, in theory, the safest/least risky portfolio (De Carvalho \emph{et
al.},
\protect\hyperlink{ref-rawl2012}{2012}\protect\hyperlink{ref-rawl2012}{b}).
Its primary input is a variance covariance matrix, which it uses to
minimize aggregate portfolio volatility. This is accomplished by
over-weighting low volatility and low correlation securities (De
Carvalho \emph{et al.},
\protect\hyperlink{ref-rawl2012}{2012}\protect\hyperlink{ref-rawl2012}{b}).

Let \(\sum\) indicate the markets variance covariance matrix and
\(w=\{w_i,..., w_N \}\) be a vector of length N containing individual
security weights. The vector containing the MV portfolio's weights can
now be described as (De Carvalho \emph{et al.},
\protect\hyperlink{ref-rawl2012}{2012}\protect\hyperlink{ref-rawl2012}{b}):

\begin{center}
$w^*=arg\min(w'\sum w)\ \ \ s.t.\ \sum^N_iw_i=1$ 
\end{center}

This approach often works well out of sample, but if left unrestricted
is known to build highly concentrated portfolio's (De Prado,
\protect\hyperlink{ref-lopez}{2016}). Its sole objective to minimize
portfolio volatility is likely the the primary reason for this. When
near the trough of its objective function it to achieves minor
reductions in \emph{ex ante} volatility by greatly favoring a small
number of low volatility/correlation securities (De Prado,
\protect\hyperlink{ref-lopez}{2016}: 68){]}. This tendency to produce
highly concentrated portfolio's can be costly out of sample as the
portfolio does not sufficiently diversify its idiosyncratic risk. It
puts too many eggs in too few baskets. In practice This issue can be
countered by applying cleaver maximum and minimum portfolio weight
constraints.

\hypertarget{inverse-varience-iv-weighting}{%
\paragraph{Inverse-Varience (IV)
Weighting}\label{inverse-varience-iv-weighting}}

The IV portfolio, referred to as the equal-risk budget (ERB) portfolio
in De Carvalho \emph{et al.}
(\protect\hyperlink{ref-leote}{2012}\protect\hyperlink{ref-leote}{a}),
aims to allocate an equal risk budget to each investible security (De
Carvalho \emph{et al.},
\protect\hyperlink{ref-leote}{2012}\protect\hyperlink{ref-leote}{a}).
Where the risk budget is defined as the the product of a security's
weight and volatility. Therefore, if we define \(\sigma_i\) as security
i's volatility, then risk buckets are equally distributed across N
securities by setting security weights as such:

\begin{center} 
$w_{iv}=(\frac{1/\sigma_1}{\sum^N_{j=1} 1/\sigma}, ...,\frac{1/\sigma_N}{\sum^N_{j=1} 1/\sigma} )$ 
\end{center}

By this definition each portfolio's weight is proportional to the
inverse of its variance (hence its name). The IV portfolio's weighting
strategy assumes that adequate diversification is attained when
allocating capital according to individual security variances and
thereby ignores the role of co-variations between securities on
portfolio volatility.

De Carvalho \emph{et al.}
(\protect\hyperlink{ref-leote}{2012}\protect\hyperlink{ref-leote}{a})
found that, if all securities posses the same sharp ratio and
correlation coefficients between each security are equal, then the IV
portfolio is efficient from a mean variance stand point and obtains the
highest possible sharp ratio.

\hypertarget{equal-risk-contribution-erc}{%
\paragraph{Equal Risk Contribution
(ERC)}\label{equal-risk-contribution-erc}}

The ERC portfolio is similar to the IV, but also takes the covariance
between securities into account when balancing risk contributions (De
Carvalho \emph{et al.},
\protect\hyperlink{ref-leote}{2012}\protect\hyperlink{ref-leote}{a}).
The basic idea behind the ERC is to weight the portfolio such that each
security contributes equally to overall portfolio risk, which in turn
maximises risk diversification (Maillard,
\protect\hyperlink{ref-maillard2010}{2010}). Generally speaking the ERC
acts similar to a weight constrained MV portfolio, with constraints
ensuring that an adequate level of idiosyncratic risk is diversified.
Following Maillard (\protect\hyperlink{ref-maillard2010}{2010}), the
weights of an ERC portfolio \(x=(x_1,x_2,...,x_n)\) consisting of n
assets can be calculated as follows:

let \(\sigma_i^2\) resemble asset i's variance, \(\sigma_{ij}\) the
covariance between asset i and j and \(\sum\) be the markets variance
covariance matrix. Portfolio risk can now be written as
\(sigma(x)=\sqrt{x^T\sum x}=\sum_i\sum_{j\neq i}x_ix_j\sigma_{ij}\) and
the marginal risk contribution, \(\partial_{x_i}\sigma(x)\), can then be
defined as:

\begin{center}
$\partial_{x_i}\sigma(x)=\frac{\partial\sigma(x)}{\partial x_i}=\frac{x_i\sigma_i^2+\sum_{j\neq i}x_j\sigma_{ij}}{\sigma(x)}$ 
\end{center}

Therefore, \(\partial_{x_i}\sigma(x)\) refers to the change in portfolio
volatility resulting from a small change in asset i's weight (Maillard,
\protect\hyperlink{ref-maillard2010}{2010}). ERC uses this definition to
guide its algorithms central objective to equate the risk contribution
for each asset in the portfolio \emph{ex ante}. No closed form solution
exists describing the weigts of the ERC portfolio, however, if we define
\((\sum x)_i\) as the \(i^{th}\) row resulting from the product of
\(\sum\) with x and note that \(\partial_{x_i}\sigma(x)=(\sum x)_i\),
then the optimal weight for the long only ERC can be described as those
that satisfy the following statement:

\begin{center}
$x^*=\{x \ \epsilon[0,1]^n:\sum x_i=1, x_i \times (\sum x)_i=x_j \times (\sum x)_j \ \forall  \ i,j \}$ 
\end{center}

Maillard (\protect\hyperlink{ref-maillard2010}{2010}) proved
mathematically that the ERC portfolio's \emph{ex ante} volatility is
always some where between those of the EW and MV portfolio's. De
Carvalho \emph{et al.}
(\protect\hyperlink{ref-leote}{2012}\protect\hyperlink{ref-leote}{a})
found that, if all securities posses the same sharp ratio , then the ERC
and ERB have identical portfolio weights. If in addition the correlation
coefficients between all securities are equal, then the ERC and ERB
merge into the EW portfolio and each are mean variance efficient with
the maximum attainable sharp ratio (De Carvalho \emph{et al.},
\protect\hyperlink{ref-leote}{2012}\protect\hyperlink{ref-leote}{a}).

\hypertarget{maximum-diversification-md}{%
\paragraph{Maximum Diversification
(MD)}\label{maximum-diversification-md}}

Choueifaty \& Coignard (\protect\hyperlink{ref-choueifaty2008}{2008})
originally designed the MD portfolio to maximize some diversification
ratio (DR), which he defined as the sum of each securities risk bucket
divided by portfolio volatility (De Carvalho \emph{et al.},
\protect\hyperlink{ref-leote}{2012}\protect\hyperlink{ref-leote}{a}). If
we define \(w=(w_1,...w_N)^T\) as a vector of portfolio weights, V as a
vector of asset volatilities and \(\sum\) as the covariance matrix. Then
the DR can be expresses as:

\begin{center} 
$DR= \frac{w'.V}{\sqrt{w'Vw}}$ 
\end{center}

Much like the IV and ERC portfolios, the MD portfolio attempts to
diversify its portfolio by allocating equal risk to each security
(Choueifaty \& Coignard, \protect\hyperlink{ref-choueifaty2008}{2008}).
The MD portfolio accomplishes this by over-weighting low volatility
securities and those that are less correlated (De Carvalho \emph{et
al.},
\protect\hyperlink{ref-leote}{2012}\protect\hyperlink{ref-leote}{a}).
For further detail regarding the theoretical results and properties of
the MD portfolio see Choueifaty \& Coignard
(\protect\hyperlink{ref-choueifaty2008}{2008}: 33--35).

\hypertarget{empirical-backtests-and-monte-carlo-findings}{%
\subsection{Empirical Backtests and Monte Carlo
Findings}\label{empirical-backtests-and-monte-carlo-findings}}

Choueifaty \emph{et al.} (\protect\hyperlink{ref-choueifaty2013}{2013})
conducted an empirical back test comparing the relative performance of
numerous portfolio optimisers between 1999 and 2010. They used
historical data from the MSCI world index and considered the largest
50\% of assets at each semi-annual rebalance date. To reduce the noise
in estimation, at each rebalance date covariance matrices were estimated
using the previous years worth of data (Choueifaty \emph{et al.},
\protect\hyperlink{ref-choueifaty2013}{2013}). These were then used as
the primary inputs in estimating the respective long-only portfolio
weights. The MV portfolio achieved an annual return of 6.7\% and
outperformed the ERC and EW portfolio's who returned 6.3\% and 5.8\%
respectively. Unsurprisingly, the MV portfolio possessed the lowest
daily volatility (10\%) followed by the ERC and then the EW portfolio's
(with 12.9\% and 16.4\% respectively). Accordingly the MV portfolio
scored the highest sharp ratio (0.36) followed by the ERC and EW
portfolio's (0.24 and 0.16 respectively). According to De Carvalho
\emph{et al.}
(\protect\hyperlink{ref-leote}{2012}\protect\hyperlink{ref-leote}{a})
the performance of the EW portfolio primarily depends on the premium on
small-capitalization stocks, thereby suggesting that the relatively poor
performance of the EW portfolio in Choueifaty \emph{et al.}
(\protect\hyperlink{ref-choueifaty2013}{2013}), can be attributed to the
relatively poor returns achieved by the smaller stocks in the MSCI world
index.

Despite the simplistic nature of the EW portfolio empirical studies,
like those by DeMiguel \emph{et al.}
(\protect\hyperlink{ref-demiguel2009}{2009}), who compared the EW to the
mean-variance, MV and Bayes-Stein portfolios, tend to find a
statistically insignificant difference in Sharp ratio between the naive
portfolio and those of the more advanced portfolio optimisers. In this
study the EW also performed surprisingly well from a total return
perspective.

Due to the aforementioned issues surrounding estimation error in a
market's covariance matrix Ardia \emph{et al.}
(\protect\hyperlink{ref-ardia2017}{2017}) set out to evaluate the impact
of covariance matrix misspecification on the properties of risk based
portfolio's. They used Monte Carlo methods to build six distinctive
investment universes, each with a unique, variance/correlation structure
and a varying number of assets. Numerous covariance matrix estimation
techniques were then estimated on the simulated data, one of which
serving as the benchmark. They then used the simulated data and the
various covariance matrices to access the impact of alternative
covariance specifications on the performance of the MV, IV, ERC and MD
portfolio's. The ERC and IV portfolios were found to be ``relatively
robust to covariance misspecification'', the MV was found to be
sensitive to misspecification in both the variance and covariance and
the MD portfolio was found to be robust to misspecification in the
variances but sensitive to misspecification in the covariances (Ardia
\emph{et al.}, \protect\hyperlink{ref-ardia2017}{2017}: 1).

\hypertarget{monte-carlo-methods-in-portfolio-optimisation}{%
\subsection{Monte Carlo Methods in Portfolio
Optimisation}\label{monte-carlo-methods-in-portfolio-optimisation}}

Ever since the pioneering age of computers people have shown a keen
interest in leveraging their ability to perform rapid calculations to
conduct randomized experiments (Kroese \emph{et al.},
\protect\hyperlink{ref-kroese2014}{2014}: 1). The core of Monte Carlo
simulation is in the creation of random objects and/or processes using a
computer. There are a number of reasons for doing this, but the primary
one used in this work and thereby discussed in this review is of the
sampling kind (Kroese \emph{et al.},
\protect\hyperlink{ref-kroese2014}{2014}). This typically involves the
modeling of some stochastic object or process, followed by sampling from
some probability distribution and the manipulating said sample through
some deterministic process such that the result mimics the true
underlying process. The primary idea behind Monte Carlo simulation is to
repeat this simulation process many times so that interesting properties
can be uncovered through the law of large numbers and central limit
theorem.

A financial application of this can be found in Wang \emph{et al.}
(\protect\hyperlink{ref-wang2012}{2012}) who designed a Monte Calro
procedure that (1) models both the time-series and cross-section
properties of financial market returns, this involves the estimation of
a random term's probability distribution function (pdf) using extreme
value theory. And (2) sampling from the modeled process to produce an
ensemble of market returns, with each exerting the same risk properties.
The simulated data can then be used in risk management and/or the
pricing of financial securities (Kroese \emph{et al.},
\protect\hyperlink{ref-kroese2014}{2014}; Wang \emph{et al.},
\protect\hyperlink{ref-wang2012}{2012}). This unique ability to generate
a large number of counterfactuals for an asset market with a known risk
structure has made it a uniquely powerful tool in accessing the
properties of portfolio optimization algorithms (Bailey \& Lopez De
Prado, \protect\hyperlink{ref-lopez2012}{2012}).

Glasserman (\protect\hyperlink{ref-glasserman2013}{2013}) is a useful
source for understanding the methods and applications of Monte Carlo
methods in finance.

\hypertarget{methadology}{%
\section{\texorpdfstring{Methadology
\label{methadology}}{Methadology }}\label{methadology}}

This work used Monte Carlo simulation methods to investigate the link
between a markets correlation structure and the relative performance of
the EW, MV, IV, ERC and MD portfolios. To avoid possible confusion note
the following terminology:

\begin{itemize}
\tightlist
\item
  The term market refers to a set daily returns for a number of assets.
  e.g.~the daily returns for each of the JSE ALSI constitutes between 1
  January 2019 and 1 January 2020. Since this is a Monte Carlo study,
  thousands of markets are simulated, they can therefore be thought of
  as a single observation or realization.
\item
  The term market type refers to a set or ensemble of simulated markets
  each with the same specified risk characteristics. In this study only
  the correlation structure differs between market types.
\end{itemize}

The R package MCmarket was used to simulate five distinctive market
types, each corresponding to a unique correlation structure (Potgieter,
\protect\hyperlink{ref-MCmarket}{2020}). Four of the correlation
matrices were designed \emph{ad hoc}, with each possessing a unique
correlation structure, while the fifth was estimated using S\&P 500
data. These correlation matrices range from one exhibiting no
correlation (i.e.~a diagonal matrix) to one with hierarchical clustering
(see \ref{corr_struc}).

The long only EW, MV, IV, ERC and MD portfolios were then back tested on
the simulated markets and portfolio analytics were performed on the
portfolio returns. \textbf{These portfolio analytics include the
standard deviation (sd) of daily returns, downside deviation, value at
risk (VaR), conditional VAR (CVaR), Sharp ratio, average drawdown and
maximum drawdown}.

Finally, the portfolio metrics are compared within market types across
portfolios and within portfolios, across market types.

\hypertarget{correlation-structures}{%
\subsection{\texorpdfstring{Correlation Structures
\label{corr_struc}}{Correlation Structures }}\label{correlation-structures}}

This section describes the composition and attributes behind the four
\emph{ad hoc} correlation matrices used in this study (section
\ref{adhoc}) and then the methodology behind the estimation of the
empirical correlation matrix (section \textbackslash ref\{emp) is
discussed. Finally each of the matrices top 10 eigenvalues are listed in
Table \ref{eigens}.

Note that each of the five correlation matrices described in this
section are used in separate Monte Carlo simulations, where all other
risk attributes remain constant across market types.

\hypertarget{ad-hoc}{%
\subsubsection{\texorpdfstring{Ad Hoc
\label{adhoc}}{Ad Hoc }}\label{ad-hoc}}

This section describes the four \emph{ad hoc} 50 by 50 correlation
matrices used as the key inputs in their respective Monte Carlo
simulations. See Figure \ref{corr_mats} for a graphical representation
of each correlation matrix. Note that the \emph{gen\_corr} function from
the R package \emph{MCmarket} was used in the construction of the four
\emph{ad hoc} matrices (Potgieter,
\protect\hyperlink{ref-MCmarket}{2020}).

The first and most simplistic of the four matrices is a diagonal matrix
(see Diagonal Matrix in Figure \ref{corr_mats}). It describes a market
with a zero correlation coefficient between each asset. Each of its 50
eigenvalues are equal to 1 (Table \ref{eigens}), it has no risk clusters
and is therefore plenty scope for diversification. The Monte Carlo data
set constructed using this matrix will be referred to as Market 1.

The second matrix (labeled No Clusters in Figure \ref{corr_mats}) has no
risk clusters but describes a market with significant correlation
between its constituents. Each asset has a correlation of 0.9 with is
closest neighbor (i.e.~Asset 1 and 2, 5 and 6 and 11 and 12 each have a
pairwise correlation coefficient of 0.9). Correlations then diminish
exponentially by the absolute distance between the two assets (i.e.~the
correlation between Asset 1 and 5 is \(0.9^{|1-5|}=0.6561\)). Its has a
large first eigenvalue of 15.93, but they quickly diminish such that its
9th largest eigenvalue is less than 1 at 0.79 (Table \ref{eigens}). The
Monte Carlo data set constructed using this matrix will be referred to
as Market 2.

The third matrix (labeled Five Clusters in Figure \ref{corr_mats})
contains five distinctive non-overlapping risk clusters. Assets within
the same cluster have a pairwise correlation coefficient of 0.6 while
those that are not in the same cluster are uncorrelated. Its first five
eigenvalues are 6.5, with the remaining 45 equal to 0.4 (Table
\ref{eigens}). The Monte Carlo data set constructed using this matrix
will be referred to as Market 3.

The final \emph{ad hoc} correlation matrix has three layers of
overlapping risk clusters. The first layer has 10 distinctive clusters,
within which assets have a correlation coefficient of 0.7. The second
layer has four clusters where assets that are not in same first layer
cluster have a correlation coefficient of 0.5. Assets that are in the
same third layer cluster but not clustered in layers one and two have a
correlation coefficient of 0.3. Finally, those who do not share any
cluster have a correlation coefficient of 0.05. Its largest eigenvalue
is 14.36, but they diminish fairly quickly as its third largest is only
3.3 (Table \ref{eigens}). The Monte Carlo data set constructed using
this matrix will be referred to as Market 4.

See Table \ref{eigens} for a list of each correlation matrices largest
ten eigenvalues.

\begin{figure}
\centering
\includegraphics{Thesis_files/figure-latex/corr mats-1.pdf}
\caption{\label{corr_mats} Correlation Matricies}
\end{figure}

\hypertarget{emperical}{%
\subsubsection{\texorpdfstring{Emperical
\label{emp}}{Emperical }}\label{emperical}}

The empirical correlation matrix used in this study was estimated from
the daily returns of a random subset of 50 of the largest (by market
capitalization) 100 S\&P 500 stocks between 1 January 2016 and 1 January
2021. The market capitalizations were measured as of 12 January 2020.
The covariance matrix was then calculated using the \emph{fit\_mvt}
function from the R package \emph{fitHeavyTail}. This function uses
maximum likelihood estimation and generalized expectation maximization
method to fit a multivariate t-distribution to a matrix of asset returns
(Liu \& Rubin, \protect\hyperlink{ref-liu1995}{1995}).

The estimated multivariate t distribution was found to have 4.43 degrees
of freedom and a correlation matrix shown in Figure \ref{corr_emp}. Note
that the assets were ordered by hierarchical clustering so the reader
can easily visualize the risk clusters. The correlation matrix's largest
eigenvalue is 18.6 and they quickly diminish to below zero by its 8th
largest eigenvalue (Table \ref{eigens}). The Monte Carlo data set
constructed using this matrix will be referred to as Market 5.

\begin{figure}
\centering
\includegraphics{Thesis_files/figure-latex/unnamed-chunk-1-1.pdf}
\caption{\label{corr_emp} Emperical Correlation Matrix}
\end{figure}

\begin{table}[!htbp] \centering 
  \caption{Eigenvalues} 
  \label{eigens} 
\begin{tabular}{@{\extracolsep{5pt}} ccccc} 
\\[-1.8ex]\hline 
\hline \\[-1.8ex] 
Diagonal & No Clusters & Five Clusters & Overlapping Clusters & Emperical \\ 
\hline \\[-1.8ex] 
1 & 15.93 & 6.4 & 14.36 & 18.6 \\ 
1 & 10.38 & 6.4 & 6.89 & 3.09 \\ 
1 & 6.28 & 6.4 & 3.3 & 2.52 \\ 
1 & 3.92 & 6.4 & 3.3 & 1.4 \\ 
1 & 2.59 & 6.4 & 2.49 & 1.28 \\ 
1 & 1.81 & 0.4 & 2.46 & 1.21 \\ 
1 & 1.32 & 0.4 & 1.3 & 1.03 \\ 
1 & 1.01 & 0.4 & 1.3 & 0.92 \\ 
1 & 0.79 & 0.4 & 1.3 & 0.88 \\ 
1 & 0.64 & 0.4 & 1.3 & 0.83 \\ 
\hline \\[-1.8ex] 
\end{tabular} 
\end{table}

\hypertarget{monte-carlo}{%
\subsection{Monte Carlo}\label{monte-carlo}}

This section outlines the methodology behind the Monte Carlo simulation
performed as part of this study.

A generalized version of the Monte Carlo procedure developed in Wang
\emph{et al.} (\protect\hyperlink{ref-wang2012}{2012}) was used to
simulate the five market types. This framework was build into the R
package \emph{MCmarket} which was used to conduct this project's Monte
Carlo simulations (Potgieter, \protect\hyperlink{ref-MCmarket}{2020}).
The following briefly describes this process:

An Elliptical t copula with 4.5 degrees of freedom is used, in
conjunction with a 50 by 50 correlation matrix (Section \ref{corrs}), to
simulate 300 random uniformly distributed draws (corresponding to 300
trading days) across the 50 assets. The simulated series each adhere to
the correlation structure described in their respicetive correlation
matrix. These uniformly distributed observations were then transformed
into normally distributed observations, via the inverted normal
cumulative distribution function (Potgieter,
\protect\hyperlink{ref-MCmarket}{2020}; Wang \emph{et al.},
\protect\hyperlink{ref-wang2012}{2012}: 3).

The expected returns and standard deviations of these simulated
variables were set via empirical return data from a random subset of 50
of the largest 100 S\&P500 stocks (discussed in Section \ref{emp_corr})
between 1 January 2020 and 1 January 2021. Maximum likelihood estimation
was used to fit the multivariate t distribution to the return series
using the method developed by Liu \& Rubin
(\protect\hyperlink{ref-liu1995}{1995}). This produced a series of
estimated means and variances which were used to calibrate the expected
returns and standard deviations of the simulated variables.

This process was repeated 10000 times for each of the five correlation
matrices/ market types set out in section \ref{corr_struc}. Thereby
creating 5 data sets each containing 10000 markets with 50 assets and
300 periods.

\hypertarget{back-tests} is also
applied to prevent some portfolios from building unreasonably highly
concentrated holdings, while remaining flexible enough to punish those
who do so. Therefore, these constraints act to provide a fair playing
ground for the portfolio's to compete. The back testing procedure works
as follows:

The first 100 periods are used to estimate the a covariance matrix using
the maximum likelihood methodology described in Liu \& Rubin
(\protect\hyperlink{ref-liu1995}{1995}). Interestingly, the assumption
regarding the returns adhering to the multivariate t distribution is
correct by definition since this is the distribution used in the
simulations. This covariance matrix is then used as the sole input when
calculating the weights for the respective risk-based portfolios. The
portfolio's then hold these weights over the next 50 periods, when they
are rebalanced by looking back 100 periods, calculating the covariance
matrix and the new portfolio returns. This process is repeated until all
periods in the data set are exhausted. Since there are 300 periods in
each market, each portfolio is weighted four times and 200 periods of
daily returns are calculated for each portfolio.

\hypertarget{portfolio-analytics}{%
\subsection{Portfolio Analytics}\label{portfolio-analytics}}

This section describes the portfolio performance, risk and concentration
metrics used to evaluate and compare portfolio performance. The Sharp
ratio is used to evaluate risk adjusted return, standard deviation,
downside deviation and value at risk are used to access portfolio risk.
Finally, the effective number of constitutes, calculated as the inverse
of the Herfindahl-Hirschman index (HHI), and the effective number of
bets, calculated following Meucci
(\protect\hyperlink{ref-meucci2010}{2010}), are calculated to compare
portfolio concentration (Rhoades,
\protect\hyperlink{ref-rhoades1993}{1993}).

Since Markowitz (\protect\hyperlink{ref-markowitz}{1952}) variance of
returns has been the standard measure for risk in the financial industry
({\textbf{???}}). With the standard deviation simply being the
square-root of the variance, it too is widely used. Standard deviation
also benefits due to its relative ease in interpretation. Standard
deviation is also key in calculating the next two portfolio performance
metrics described in this study, namely the Sharp ratio and value at
risk (VaR).

The Sharp ratio is a measure of a portfolio's risk adjusted returns.
Generally speaking, the Sharp ratio is calculated by dividing the
portfolio return by some measure of portfolio risk, it is therefore
interpreted as the return per unit of risk. In this work standard
deviation is used as the measure of risk.

The 95\% VaR is another risk metric used to evaluate portfolio risk
performance in this study. It is one of the financial industry standards
for measures for downside risk and can be interpreted as the maximum
return expected from in the worst 5\% of scenarios (Peterson \& Carl,
\protect\hyperlink{ref-PerformanceAnalytics}{2020}). That is, in the
worst 5\% of scenarios, one should expect to loose at least this amount.
The particular version of VaR used here is the Gaussian VaR, which is
calculated by assuming that returns are normally \(N(\mu,\sigma)\)
distributed, where \(\mu\) and \(\sigma\) are estimated using historical
data. The probability distribution assumptions allows one to attach a
probability values to possible future portfolio returns. This assumption
can be dangerous in practice, however in this study it correct by
definition as the return series were each simulated to be normally
distributed. It should therefore result in an accurate estimate of
downside risk.

The HHI estimates portfolio concentration by by summing the the
portfolio weights squared (Rhoades,
\protect\hyperlink{ref-rhoades1993}{1993}). A portfolio with small
weights allocated evenly across a large number of securities will have
an HHI of approximately zero, while a portfolio with all its capital
invested in a single security will have the maximum HHI of 10000. The
effective number of constitutes (ENC) can then be calculated as the
inverse of the HHI, where an equally weighted portfolio with have an ENC
equal to the number of securities and more concentrated portfolio's will
have a ENC less than the number of securities. Weight based measures of
portfolio diversification are severely limited in that they are
oblivious to covariation between portfolio components. The ENC can
therefore be misleading in financial applications where portfolio
components are known to exhibit significant dependence.

Meucci (\protect\hyperlink{ref-meucci2010}{2010}) attempted to rectify
this issue when he introduced a new method to evaluate portfolio
diversification that considers the portfolio's risk structure. He used a
principle component (PC) approach to estimate the total number of
orthogonal bets in a portfolio, which he simply referred to as principle
portfolios. With this he estimated a portfolio diversification
distribution as the percentage of total portfolio variation attributed
to each principle portfolio. The effective number of orthogonal bets
(ENB) can then be calculated as the dispersion of the diversification
distribution (Meucci, \protect\hyperlink{ref-meucci2010}{2010}: 10). \#
Meucci (2009) uses a PC based approach - where if each principal
component is loaded with the same weight, we have optimal
diversification.

\begin{quote}
The maximum drawdown is calculated by first, calculating portfolio
cumulative returns and the maximum cumulative return achieved. The
maximum drawdown is then the maximum amount that the cumulative return
dips below its maximum, it is measured as a percentage of the maximum
cumulative return (Peterson \& Carl,
\protect\hyperlink{ref-PerformanceAnalytics}{2020}).
\end{quote}

\hypertarget{results-and-discussion}{%
\section{\texorpdfstring{Results and Discussion
\label{reasults}}{Results and Discussion }}\label{results-and-discussion}}

Note that the markets simulated using the diagonal correlation matrix
described in Section \ref{adhoc} will hence forth be referred to as
Market 1. Similarly, the markets simulated using the no cluster, five
clusters, overlapping clusters and empirical correlation matrices will
be respectively referred to as Market 2, Market 3, Market 4 and Market
5. Therefore, each of the Markets 1 - 5 contain a unique correlation
structure.

\hypertarget{comparing-portfolios-within-market-types}{%
\subsection{Comparing Portfolios Within Market
Types}\label{comparing-portfolios-within-market-types}}

\hypertarget{market-1}{%
\subsubsection{Market 1}\label{market-1}}

The portfolios compared here were estimated on the markets simulated
using the diagonal correlation matrix (Figure \ref{corr_mats}). The
portfolios average Sharp ratio (Sharp), standard deviation (SD),
downside deviation and VaR across the 10 000 simulated markets are shown
in Table \ref{rm1}.

On average the EW portfolio performed the best as it achieved the
highest Sharp ratio and the lowest standard deviation, downside
deviation and VaR. The IV portfolio was a close second, followed by the
ERC, MD and MV portfolios.

With such a market structure it is reasonable to expect that, depending
if the asset variance allocation is fairly evenly distribution then the
EW, MV, IV, ERC and MD will produce fairly similar.

\begin{table}[!htbp] \centering 
  \caption{Market 1 Risk Metrics} 
  \label{rm1} 
\begin{tabular}{@{\extracolsep{5pt}} cccccc} 
\\[-1.8ex]\hline 
\hline \\[-1.8ex] 
Metric & EW & MV & IV & ERC & MD \\ 
\hline \\[-1.8ex] 
Sharp & 0.19353 & 0.13501 & 0.18341 & 0.15876 & 0.14284 \\ 
SD & 0.00386 & 0.005 & 0.00386 & 0.00447 & 0.00494 \\ 
Downside Deviation & 0.00235 & 0.00318 & 0.00236 & 0.00279 & 0.00312 \\ 
VaR & -0.00545 & -0.00735 & -0.00548 & -0.00645 & -0.00722 \\ 
\hline \\[-1.8ex] 
\end{tabular} 
\end{table}

\begin{table}[!htbp] \centering 
  \caption{Market 1 Portfolio Entropy Metrics} 
  \label{em1} 
\begin{tabular}{@{\extracolsep{5pt}} cccccc} 
\\[-1.8ex]\hline 
\hline \\[-1.8ex] 
Metric & EW & MV & IV & ERC & MD \\ 
\hline \\[-1.8ex] 
ENC & 50 & 25.3 & 47.1 & 30.1 & 29.1 \\ 
ENB & 14.8 & 17.9 & 16.6 & 16.7 & 17.5 \\ 
\hline \\[-1.8ex] 
\end{tabular} 
\end{table}

\hypertarget{market-2}{%
\subsubsection{Market 2}\label{market-2}}

The portfolios compared here were estimated on the markets simulated
using the no clusters correlation matrix (Figure \ref{corr_mats}). The
portfolios average Sharp ratio (Sharp), standard deviation (SD),
downside deviation and VaR across the 10 000 simulated markets are shown
in Table \ref{rm2}.

On average the EW, MV, IV and ERC portfolios performed very similarly
according to the standard deviation, downside deviation and VaR metrics.
Out of these four portfolios the ERC attained the highest sharp ratio
and narrowly achieved the lowest scores across three risk measures.
Despite performing the worst from a risk perspective, the MD portfolio
attained the highest overall Sharp ratio. Thereby indicating that the MD
managed to attain significantly higher average returns compared to the
other portfolio's. On the other hand, despite performing well from a
risk perspective, the MV portfolio attained a Sharp ratio significantly
lower than the other portfolios

\begin{table}[!htbp] \centering 
  \caption{Market 2 Risk Metrics} 
  \label{rm2} 
\begin{tabular}{@{\extracolsep{5pt}} cccccc} 
\\[-1.8ex]\hline 
\hline \\[-1.8ex] 
Metric & EW & MV & IV & ERC & MD \\ 
\hline \\[-1.8ex] 
Sharp & 0.05171 & 0.03482 & 0.04955 & 0.05312 & 0.06388 \\ 
SD & 0.01451 & 0.01434 & 0.01436 & 0.01431 & 0.0156 \\ 
Downside Deviation & 0.00983 & 0.00985 & 0.00974 & 0.00968 & 0.01047 \\ 
VaR & -0.02273 & -0.02273 & -0.02252 & -0.02238 & -0.02424 \\ 
\hline \\[-1.8ex] 
\end{tabular} 
\end{table}

\hypertarget{market-3}{%
\subsubsection{Market 3}\label{market-3}}

The portfolios compared here were estimated on the markets simulated
using the five clusters correlation matrix (Figure \ref{corr_mats}). The
portfolios average Sharp ratio (Sharp), standard deviation (SD),
downside deviation and VaR across the 10 000 simulated markets are shown
in Table \ref{rm3}.

Despite there being significant and distinct risk clusters in the
markets

\begin{table}[!htbp] \centering 
  \caption{Market 3 Risk Metrics} 
  \label{rm3} 
\begin{tabular}{@{\extracolsep{5pt}} cccccc} 
\\[-1.8ex]\hline 
\hline \\[-1.8ex] 
Metric & EW & MV & IV & ERC & MD \\ 
\hline \\[-1.8ex] 
Sharp & 0.08001 & 0.04974 & 0.07655 & 0.07605 & 0.06638 \\ 
SD & 0.0094 & 0.01049 & 0.0093 & 0.00935 & 0.01065 \\ 
Downside Deviation & 0.00623 & 0.00712 & 0.00618 & 0.00622 & 0.00714 \\ 
VaR & -0.01437 & -0.01639 & -0.01425 & -0.01433 & -0.01646 \\ 
\hline \\[-1.8ex] 
\end{tabular} 
\end{table}

\hypertarget{market-4}{%
\subsubsection{Market 4}\label{market-4}}

\begin{table}[!htbp] \centering 
  \caption{Market 4 Risk Metrics} 
  \label{eigens} 
\begin{tabular}{@{\extracolsep{5pt}} cccccc} 
\\[-1.8ex]\hline 
\hline \\[-1.8ex] 
Metric & EW & MV & IV & ERC & MD \\ 
\hline \\[-1.8ex] 
Sharp & 0.05409 & 0.03229 & 0.05195 & 0.0518 & 0.04758 \\ 
SD & 0.01395 & 0.01456 & 0.01378 & 0.01379 & 0.01486 \\ 
Downside Deviation & 0.00944 & 0.01002 & 0.00933 & 0.00934 & 0.0101 \\ 
VaR & -0.02181 & -0.02312 & -0.02156 & -0.02159 & -0.02334 \\ 
\hline \\[-1.8ex] 
\end{tabular} 
\end{table}

\hypertarget{market-5}{%
\subsubsection{Market 5}\label{market-5}}

\begin{table}[!htbp] \centering 
  \caption{Market 5 Risk Metrics} 
  \label{eigens} 
\begin{tabular}{@{\extracolsep{5pt}} cccccc} 
\\[-1.8ex]\hline 
\hline \\[-1.8ex] 
Metric & EW & MV & IV & ERC & MD \\ 
\hline \\[-1.8ex] 
Sharp & 0.05025 & 0.0396 & 0.0478 & 0.04794 & 0.0442 \\ 
SD & 0.01508 & 0.0176 & 0.01504 & 0.01531 & 0.01746 \\ 
Downside Deviation & 0.01023 & 0.01204 & 0.01022 & 0.01041 & 0.0119 \\ 
VaR & -0.02371 & -0.02795 & -0.02368 & -0.02412 & -0.02764 \\ 
\hline \\[-1.8ex] 
\end{tabular} 
\end{table}

\hypertarget{comparing-portfolios-across-market-types}{%
\subsection{Comparing Portfolios Across Market
Types}\label{comparing-portfolios-across-market-types}}

\hypertarget{naive}{%
\subsubsection{Naive}\label{naive}}

\hypertarget{minimum-variance}{%
\subsubsection{Minimum Variance}\label{minimum-variance}}

\hypertarget{inverse-volatility}{%
\subsubsection{Inverse Volatility}\label{inverse-volatility}}

\hypertarget{equal-risk-contribution}{%
\subsubsection{Equal Risk Contribution}\label{equal-risk-contribution}}

\hypertarget{maximum-diversification}{%
\subsubsection{Maximum Diversification}\label{maximum-diversification}}

\hypertarget{discussion}{%
\subsection{Discussion}\label{discussion}}

\hypertarget{conclusion}{%
\section{\texorpdfstring{Conclusion
\label{conclusion}}{Conclusion }}\label{conclusion}}

I hope you find this template useful. Remember, stackoverflow is your
friend - use it to find answers to questions. Feel free to write me a
mail if you have any questions regarding the use of this package. To
cite this package, simply type citation(``Texevier'') in Rstudio to get
the citation for Katzke (\protect\hyperlink{ref-Texevier}{2017}) (Note
that united references in your bibtex file will not be included in
References).

\newpage

\hypertarget{references}{%
\section*{References}\label{references}}
\addcontentsline{toc}{section}{References}

\hypertarget{refs}{}
\leavevmode\hypertarget{ref-ardia2017}{}%
Ardia, D., Bolliger, G., Boudt, K. \& Gagnon-Fleury, J.-P. 2017. The
impact of covariance misspecification in risk-based portfolios.
\emph{Annals of Operations Research}. 254(1-2):1--16.

\leavevmode\hypertarget{ref-lopez2012}{}%
Bailey, D.H. \& Lopez De Prado, M. 2012. Balanced baskets: A new
approach to trading and hedging risks. \emph{Journal of Investment
Strategies (Risk Journals)}. 1(4).

\leavevmode\hypertarget{ref-Matrix}{}%
Bates, D. \& Maechler, M. 2019. \emph{Matrix: Sparse and dense matrix
classes and methods}. ed. {[}Online{]}, Available:
\url{https://CRAN.R-project.org/package=Matrix}.

\leavevmode\hypertarget{ref-choueifaty2008}{}%
Choueifaty, Y. \& Coignard, Y. 2008. Toward maximum diversification.
\emph{The Journal of Portfolio Management}. 35(1):40--51.

\leavevmode\hypertarget{ref-choueifaty2013}{}%
Choueifaty, Y., Froidure, T. \& Reynier, J. 2013. Properties of the most
diversified portfolio. \emph{Journal of investment strategies}.
2(2):49--70.

\leavevmode\hypertarget{ref-clarke2011}{}%
Clarke, R., De Silva, H. \& Thorley, S. 2011. Minimum-variance portfolio
composition. \emph{The Journal of Portfolio Management}. 37(2):31--45.

\leavevmode\hypertarget{ref-leote}{}%
De Carvalho, R.L., Lu, X. \& Moulin, P. 2012a. Demystifying equity
risk--based strategies: A simple alpha plus beta description. \emph{The
Journal of Portfolio Management}. 38(3):56--70.

\leavevmode\hypertarget{ref-rawl2012}{}%
De Carvalho, R.L., Lu, X. \& Moulin, P. 2012b. Demystifying equity
risk--based strategies: A simple alpha plus beta description. \emph{The
Journal of Portfolio Management}. 38(3):56--70.

\leavevmode\hypertarget{ref-demiguel2009}{}%
DeMiguel, V., Garlappi, L. \& Uppal, R. 2009. Optimal versus naive
diversification: How inefficient is the 1/n portfolio strategy?
\emph{The review of Financial studies}. 22(5):1915--1953.

\leavevmode\hypertarget{ref-lopez}{}%
De Prado, M.L. 2016. Building diversified portfolios that outperform out
of sample. \emph{The Journal of Portfolio Management}. 42(4):59--69.

\leavevmode\hypertarget{ref-fama1992}{}%
Fama, E.F. \& French, K.R. 1992. The cross-section of expected stock
returns. \emph{the Journal of Finance}. 47(2):427--465.

\leavevmode\hypertarget{ref-glasserman2013}{}%
Glasserman, P. 2013. \emph{Monte carlo methods in financial
engineering}. ed. Vol. 53. Springer Science \& Business Media.

\leavevmode\hypertarget{ref-higham2002}{}%
Higham, N.J. 2002. Computing the nearest correlation matrix---a problem
from finance. \emph{IMA journal of Numerical Analysis}. 22(3):329--343.

\leavevmode\hypertarget{ref-Texevier}{}%
Katzke, N.F. 2017. \emph{Texevier: Package to create elsevier templates
for rmarkdown}. ed. Stellenbosch, South Africa: Bureau for Economic
Research.

\leavevmode\hypertarget{ref-kroese2014}{}%
Kroese, D.P., Brereton, T., Taimre, T. \& Botev, Z.I. 2014. Why the
monte carlo method is so important today. \emph{Wiley Interdisciplinary
Reviews: Computational Statistics}. 6(6):386--392.

\leavevmode\hypertarget{ref-liu1995}{}%
Liu, C. \& Rubin, D.B. 1995. ML estimation of the t distribution using
em and its extensions, ecm and ecme. \emph{Statistica Sinica}. 19--39.

\leavevmode\hypertarget{ref-maillard2010}{}%
Maillard, T., Roncalli. 2010. The properties of equally weighted risk
contribution portfolios. \emph{The Journal of Portfolio Management}.
36(4):60--70.

\leavevmode\hypertarget{ref-markowitz}{}%
Markowitz, H. 1952. Portfolio selection. \emph{The Journal of Finance}.
7(1):77--91.

\leavevmode\hypertarget{ref-meucci2010}{}%
Meucci, A. 2010. Managing diversification.

\leavevmode\hypertarget{ref-PerformanceAnalytics}{}%
Peterson, B.G. \& Carl, P. 2020. \emph{PerformanceAnalytics: Econometric
tools for performance and risk analysis}. ed. {[}Online{]}, Available:
\url{https://CRAN.R-project.org/package=PerformanceAnalytics}.

\leavevmode\hypertarget{ref-MCmarket}{}%
Potgieter, N. 2020. \emph{MCmarket: Package to simplify the monte carlo
simulation of financial markets}. ed. Stellenbosch, South Africa:
Stellenbosch University.

\leavevmode\hypertarget{ref-rhoades1993}{}%
Rhoades, S.A. 1993. The herfindahl-hirschman index. \emph{Fed. Res.
Bull.} 79:188.

\leavevmode\hypertarget{ref-wang2012}{}%
Wang, P., Sullivan, R.N. \& Ge, Y. 2012. Risk-based dynamic asset
allocation withExtreme tails and correlations. \emph{The Journal of
Portfolio Management}. 38(4):26--42.

\leavevmode\hypertarget{ref-zhou2019}{}%
Zhou, R., Liu, J., Kumar, S. \& Palomar, D.P. 2019. Robust factor
analysis parameter estimation. In ed. Springer \emph{International
conference on computer aided systems theory}. 3--11.

\newpage

\hypertarget{appendix}{%
\section*{Appendix}\label{appendix}}
\addcontentsline{toc}{section}{Appendix}

\begin{table}[!htbp] \centering 
  \caption{Asset Means and Sd's} 
  \label{msd} 
\begin{tabular}{@{\extracolsep{5pt}} cccccc} 
\\[-1.8ex]\hline 
\hline \\[-1.8ex] 
Asset & Mean & Sd & Asset... & Mean... & Sd... \\ 
\hline \\[-1.8ex] 
Asset\_1 & -0.00011 & 0.03277 & Asset\_26 & 0.00247 & 0.02671 \\ 
Asset\_2 & 0.00023 & 0.02314 & Asset\_27 & -0.00132 & 0.04888 \\ 
Asset\_3 & -3e-05 & 0.02359 & Asset\_28 & 0.00085 & 0.02387 \\ 
Asset\_4 & -0.0018 & 0.03654 & Asset\_29 & 0.00024 & 0.01613 \\ 
Asset\_5 & -0.00036 & 0.02925 & Asset\_30 & 0.00086 & 0.01884 \\ 
Asset\_6 & 0.00052 & 0.02396 & Asset\_31 & 0.00118 & 0.03147 \\ 
Asset\_7 & 0.00058 & 0.02451 & Asset\_32 & 0.00053 & 0.02882 \\ 
Asset\_8 & 0.00099 & 0.02287 & Asset\_33 & 0.00166 & 0.02253 \\ 
Asset\_9 & 0.00057 & 0.02929 & Asset\_34 & 0.00179 & 0.03002 \\ 
Asset\_10 & 0.00049 & 0.01736 & Asset\_35 & 0.00235 & 0.02574 \\ 
Asset\_11 & -0.00088 & 0.01907 & Asset\_36 & 0.00157 & 0.02307 \\ 
Asset\_12 & 5e-05 & 0.0301 & Asset\_37 & 0.0017 & 0.02566 \\ 
Asset\_13 & -0.00095 & 0.03185 & Asset\_38 & 0.00157 & 0.02262 \\ 
Asset\_14 & -1e-04 & 0.02111 & Asset\_39 & 0.00054 & 0.03116 \\ 
Asset\_15 & 0.00102 & 0.02555 & Asset\_40 & 0.00242 & 0.02494 \\ 
Asset\_16 & 0.00174 & 0.02582 & Asset\_41 & 0.00246 & 0.02186 \\ 
Asset\_17 & 0.00241 & 0.03639 & Asset\_42 & 0.00113 & 0.01951 \\ 
Asset\_18 & 0.00033 & 0.03967 & Asset\_43 & 0.00181 & 0.02099 \\ 
Asset\_19 & 0.00102 & 0.02636 & Asset\_44 & -0.00045 & 0.02297 \\ 
Asset\_20 & 0.00173 & 0.02697 & Asset\_45 & 4e-05 & 0.02354 \\ 
Asset\_21 & -0.00034 & 0.01405 & Asset\_46 & 0.00485 & 0.03553 \\ 
Asset\_22 & 0.0011 & 0.0233 & Asset\_47 & 0.00064 & 0.02275 \\ 
Asset\_23 & 0.00236 & 0.02663 & Asset\_48 & 1e-04 & 0.02713 \\ 
Asset\_24 & -0.00055 & 0.03195 & Asset\_49 & 0.00071 & 0.02186 \\ 
Asset\_25 & -0.00283 & 0.03227 & Asset\_50 & -0.00203 & 0.03131 \\ 
\hline \\[-1.8ex] 
\end{tabular} 
\end{table}

\hypertarget{hierarchical-risk-parity-hrp}{%
\subsubsection{Hierarchical Risk Parity
(HRP)}\label{hierarchical-risk-parity-hrp}}

Due to the multitude of robustness issues related to traditional
portfolio optimisers, De Prado (\protect\hyperlink{ref-lopez}{2016})
developed a new approach incorporating machine-learning methods and
graph theory ({\textbf{???}}). De Prado
(\protect\hyperlink{ref-lopez}{2016}) argues that the ``lack of
hierarchical structure in a correlation matrix allows weights to vary
freely in unintended ways'' and that this contributes to the instability
issues. His HRP algorithm requires only a singular co-variance matrix
and can utilize the information within without the need for the positive
definite property (De Prado, \protect\hyperlink{ref-lopez}{2016}). This
procedure works in three stages:

De Prado (\protect\hyperlink{ref-lopez}{2016}) carried out an in sample
simulation study comparing the respective allocations of the long-only
minimum variance, IVP and HRP portfolios using a co-variance matrix
using a condition number that is ``not unfavourable'' to the minimum
variance portfolio. The simulated data consisted of 10000 observations
across 10 variables. The following findings were made: The minimum
variance portfolio concentrated 92.66\% of funds in the top 5 holdings
and assigned a zero weight to 3 assets. \emph{Conversly}, HRP only
assigned 62.5\% of its funds to the top 5 holdings (De Prado,
\protect\hyperlink{ref-lopez}{2016}). The minimum variance portfolio's
objective function causes it to build highly concentrated portfolio's in
favor of a small reduction in volatility; the HRP portfolio had only a
slightly higher volatility (De Prado,
\protect\hyperlink{ref-lopez}{2016}). This apparent diversification
advantage achieved by the minimum variance portfolio is rather deceptive
as the portfolio remains highly susceptible to idiosyncratic risk
incidents within its top holdings (De Prado,
\protect\hyperlink{ref-lopez}{2016}). This claim was further validated
by the finding that HRP achieved significantly lower out of sample
variance compared to the minimum variance portfolio.

\hypertarget{appendix-a}{%
\subsection*{Appendix A}\label{appendix-a}}
\addcontentsline{toc}{subsection}{Appendix A}

Some appendix information here

\hypertarget{appendix-b}{%
\subsection*{Appendix B}\label{appendix-b}}
\addcontentsline{toc}{subsection}{Appendix B}

\bibliography{Tex/ref}





\end{document}
